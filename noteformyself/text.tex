\section{Section name}

\subsection{Fonts in math mode}

    We use unicode-math package to support unicode math symbols, the following is a list of some common math symbols:
    \begin{itemize}
        \item $\alpha$, $\beta$, $\gamma$, $\delta$, $\epsilon$, $\zeta$, $\eta$, $\theta$, $\iota$, $\kappa$, $\lambda$, $\mu$, $\nu$, $\xi$, $\pi$, $\rho$, $\sigma$, $\tau$, $\upsilon$, $\phi$, $\chi$, $\psi$, $\omega$
        \item $\Alpha$, $\Beta$, $\Gamma$, $\Delta$, $\Epsilon$, $\Zeta$, $\Eta$, $\Theta$, $\Iota$, $\Kappa$, $\Lambda$, $\Mu$, $\Nu$, $\Xi$, $\Pi$, $\Rho$, $\Sigma$, $\Tau$, $\Upsilon$, $\Phi$, $\Chi$, $\Psi$, $\Omega$
        \item $\infty$, $\partial$, $\nabla$, $\exists$, $\forall$, $\neg$, $\wedge$, $\vee$, $\implies$, $\iff$, $\subseteq$, $\supseteq$, $\cap$, $\cup$, $\setminus$, $\emptyset$
        \item $0, 1, 2, 3, 4, 5, 6, 7, 8, 9$
        \item $+, -, \times, \div, =, <, >, \leq, \geq$
        \item $a, b, c, d, e, f, g, h, i, j, k, l, m, n, o, p, q, r, s, t, u, v, w, x, y, z$
        \item $A, B, C, D, E, F, G, H, I, J, K, L, M, N, O, P, Q, R, S, T, U, V, W, X, Y, Z$
        \item \(\cala, \calb, \calc, \cald, \cale, \calf, \calg, \calh, \cali, \calj, \calk, \call, \calm, \caln, \calo, \calp, \calq, \calr, \cals, \calt, \calu, \calv, \calw, \calx, \caly, \calz\)
        \item \(\calA, \calB, \calC, \calD, \calE, \calF, \calG, \calH, \calI, \calJ, \calK, \calL, \calM, \calN, \calO, \calP, \calQ, \calR, \calS, \calT, \calU, \calV, \calW, \calX, \calY, \calZ\)
        \item \(\bba, \bbb, \bbc, \bbd, \bbe, \bbf, \bbg, \bbh, \bbi, \bbj, \bbk, \bbl, \bbm, \bbn, \bbo, \bbp, \bbq, \bbr, \bbs, \bbt, \bbu, \bbv, \bbw, \bbx, \bby, \bbz\)
        \item \(\bbA, \bbB, \bbC, \bbD, \bbE, \bbF, \bbG, \bbH, \bbI, \bbJ, \bbK, \bbL, \bbM, \bbN, \bbO, \bbP, \bbQ, \bbR, \bbS, \bbT, \bbU, \bbV, \bbW, \bbX, \bbY, \bbZ\)
        \item \(\sfa, \sfb, \sfc, \sfd, \sfe, \sff, \sfg, \sfh, \sfi, \sfj, \sfk, \sfl, \sfm, \sfn, \sfo, \sfp, \sfq, \sfr, \sfs, \sft, \sfu, \sfv, \sfw, \sfx, \sfy, \sfz\)
        \item \(\sfA, \sfB, \sfC, \sfD, \sfE, \sfF, \sfG, \sfH, \sfI, \sfJ, \sfK, \sfL, \sfM, \sfN, \sfO, \sfP, \sfQ, \sfR, \sfS, \sfT, \sfU, \sfV, \sfW, \sfX, \sfY, \sfZ\)
        \item \(\rma, \rmb, \rmc, \rmd, \rme, \rmf, \rmg, \rmh, \rmi, \rmj, \rmk, \rml, \rmm, \rmn, \rmo, \rmp, \rmq, \rmr, \rms, \rmt, \rmu, \rmv, \rmw, \rmx, \rmy, \rmz\)
        \item \(\rmA, \rmB, \rmC, \rmD, \rmE, \rmF, \rmG, \rmH, \rmI, \rmJ, \rmK, \rmL, \rmM, \rmN, \rmO, \rmP, \rmQ, \rmR, \rmS, \rmT, \rmU, \rmV, \rmW, \rmX, \rmY, \rmZ\)
        \item \(\fraka, \frakb, \frakc, \frakd, \frake, \frakf, \frakg, \frakh, \fraki, \frakj, \frakk, \frakl, \frakm, \frakn, \frako, \frakp, \frakq, \frakr, \fraks, \frakt, \fraku, \frakv, \frakw, \frakx, \fraky, \frakz\)
        \item \(\frakA, \frakB, \frakC, \frakD, \frakE, \frakF, \frakG, \frakH, \frakI, \frakJ, \frakK, \frakL, \frakM, \frakN, \frakO, \frakP, \frakQ, \frakR, \frakS, \frakT, \frakU, \frakV, \frakW, \frakX, \frakY, \frakZ\)
        \item \(\scrA, \scrB, \scrC, \scrD, \scrE, \scrF, \scrG, \scrH, \scrI, \scrJ, \scrK, \scrL, \scrM, \scrN, \scrO, \scrP, \scrQ, \scrR, \scrS, \scrT, \scrU, \scrV, \scrW, \scrX, \scrY, \scrZ\)
        \item \(\scra, \scrb, \scrc, \scrd, \scre, \scrf, \scra, \scrh, \scri, \scrj, \scrk, \scrl, \scrm, \scrn, \scro, \scrp, \scrq, \scrr, \scrs, \scrt, \scru, \scrv, \scrw, \scrx, \scry, \scrz\)
        \item \(\bfA, \bfB, \bfC, \bfD, \bfE, \bfF, \bfG, \bfH, \bfI, \bfJ, \bfK, \bfL, \bfM, \bfN, \bfO, \bfP, \bfQ, \bfR, \bfS, \bfT, \bfU, \bfV, \bfW, \bfX, \bfY, \bfZ\)
        \item \(\bfa, \bfb, \bfc, \bfd, \bfe, \bff, \bfg, \bfh, \bfi, \bfj, \bfk, \bfl, \bfm, \bfn, \bfo, \bfp, \bfq, \bfr, \bfs, \bft, \bfu, \bfv, \bfw, \bfx, \bfy, \bfz\)
    \end{itemize}

\subsection{Theorems and definitions}
    
    There are two types of theorem environments, one is with background color, the other is without background color. 
    The following is a list of theorem environments supported by this template:

    \begin{definition}[this is a definition]
        test    
    \end{definition}
    \begin{proposition}[this is a proposition]
        test 
    \end{proposition}
    \begin{proof}
        This is a proof environment, it is used to prove theorems, propositions, lemmas, corollaries, etc.
        We allow to use step environments inside the proof environment, such as:
        \begin{step}\label{step:1_in_proof_1}
            This is a step environment, it is used to break down the proof into smaller steps.
        \end{step}
        \begin{step}\label{step:2_in_proof_1}
            This is another step environment, it is used to break down the proof into smaller steps.
        \end{step}
        And the step environment should be used inside the proof environment.
        The proof environment will automatically end with a square box.
    \end{proof}
    
    \begin{theorem}[this is a theorem]
        test 
    \end{theorem}
    \begin{proof}
        This is a proof environment.
        The step environment is labelled in the proof environment.
        A new proof environment will refresh the step environment counter.
        \begin{step}\label{step:1_in_proof_2}
            Goal 1.
        \end{step}
        Proof of Goal 1.
        \begin{step}\label{step:2_in_proof_2}
            Goal 2.
        \end{step}
        Proof of Goal 2.

        Here we test the hyperlink to the step environment \cref{step:1_in_proof_1}.

        You can also use the claim environment to make a claim in the proof environment, such as:
        \begin{claim}\label{claim:1_in_proof_2}
            This is a claim environment, it is used to make a claim in the proof environment.
        \end{claim}
        And the claim environment should be used inside the proof environment.
    \end{proof}
    \begin{proof}[Proof of \cref{claim:1_in_proof_2}]
        This is a proof for the \cref{claim:1_in_proof_2}.
    \end{proof}

    \begin{lemma}[this is a lemma]
        test
    \end{lemma}
    \begin{corollary}[this is a corollary]
        test 
    \end{corollary}
    \begin{question}[this is a question]
        test
    \end{question}
    \begin{conjecture}[this is a conjecture]
        test
    \end{conjecture}

    \begin{example}[this is an example]
        test 
    \end{example}
    \begin{exercise}[this is an exercise]
        test
    \end{exercise}
    \begin{remark}[this is a remark]
        test
    \end{remark}
    \begin{proof}[this is a proof]
        test
    \end{proof}


